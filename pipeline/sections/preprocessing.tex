\section{PREPROCESSING}
Preprocessing of sequence files should adhere to the following.

\subsection{Adapter Clipping}
Sequence files must have adapters clipped. Clipping should be performed using the FASTX Toolkit, fastx\_clipper function.

\subsubsection{Manifest Options}
The following manifest option is required. The adapter sequence is specific to the associated sequence file. If paired sequencing is performed, the pairs should contain their own adapter sequences.
\begin{verbatim}
{
  ...
  "fastx_clipper" : {
    "adapter" : string
  }
  ...
}
\end{verbatim}

\subsubsection{FASTX fastx\_clipper}
Version: 0.0.6 \\\\
The following arguments should be supplied to the FASTX Toolkit fastx\_clipper function:
\begin{verbatim}
  fastx_clipper -a {adapter} -l 0 -i {filename} -o {unique-id}.clipped

  Description of arguments:
  -a {adapter}      : Target adapter sequence for trimming.
                      Taken from the fastx_clipper::adapter manifest option.
  -l 0              : Keep all reads, even adapter-only.
  -i {file}         : Input file. Derived from the file manifest option.
  -o {name}.clipped : Output file. Derived from the name manifest option.
\end{verbatim}
